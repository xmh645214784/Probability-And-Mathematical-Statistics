\documentclass[a4paper,twocolumn]{ctexart}
\usepackage{ctex}
\usepackage{amsmath}


\graphicspath{{./figures/}}
%首行缩进两字符 利用\indent \noindent进行控制
\usepackage{indentfirst}
\setlength{\parindent}{2em}

%算法包
\usepackage{caption}
\usepackage{algorithm}
\usepackage{algorithmic}

%页边距包
\usepackage{geometry}
\geometry {left=2.0cm ,right=2.0cm,top=2.5cm,bottom=2.5cm}

%枚举
\usepackage{enumerate}

%代码
\usepackage{listings}

%算法input output
\renewcommand{\algorithmicrequire}{\textbf{Input:}} % Use Input in the format of Algorithm
\renewcommand{\algorithmicensure}{\textbf{Output:}} % Use Output in the format of Algorithm
% for fig without caption: #1: width/size; #2: fig file
\newcommand{\fignocaption}[2]{
	\begin{figure}[htp]
		\centering
		\includegraphics[#1]{#2}
	\end{figure}
}
% for fig with caption: #1: width/size; #2: fig file; #3: fig caption
\newcommand{\fig}[3]{
	\begin{figure}[htp]
		\centering
		\includegraphics[#1]{#2}
		\caption[labelInTOC]{#3}
	\end{figure}
}
%数学符号
\usepackage{amssymb}
\title{Probability And Mathmatical Statistics\\Homework 8}
\author{151220131 谢旻晖}
\date{}
\begin{document}
\maketitle
\section*{2.}
\noindent 记同时工作的终端数为$X$,则$X\sim B(120,0.05)$,$EX=6$,$DX=npq=5.7$.\\
由拉普拉斯中心极限定理,$X$近似服从$N(6,5.7)$.
\begin{align*}
P(X\ge 10)&=1-P(X<10)\\
&=1-P(\frac{X-6}{\sqrt{5.7}}\le \frac{4}{\sqrt{5.7}})\\
&=1-\Phi(1.68)\\
&=0.04750
\end{align*}
\section*{4}
\noindent\textbf{(1)}\\
记每个舍入误差为$X_i$,$X=\sum X_i$则$EX_i=0$,$DX_i=\frac{1}{12}$.\\
由独立同分布情形的中心极限定理,$X$近似服从$N(0,125)$.\\
\begin{align*}
&P(|X|\le 15)\\
&=P(-15\le X \le 15)\\
&=P(-\frac{3}{\sqrt{5}}\le \frac{X}{\sqrt{125}} \le \frac{3}{\sqrt{5}})\\
&=\Phi(\frac{3}{\sqrt{5}})-\Phi(-\frac{3}{\sqrt{5}})\\
&=2\Phi(\frac{3}{\sqrt{5}})-1\\
&=0.8198\\
&P(|X|\ge 15)=1-P(|X|\le 15)=0.1802
\end{align*}
\noindent\textbf{(2)}\\
设最多可有$n$个数相加使得误差总和的绝对值小于10的概率不小于0.96.
令$X=\sum_{i=1}^{n}X_i$.$EX=0$,$DX=\frac{n}{12}$.
由中心极限定理,$X\sim^{\text{近似}}N(0,\frac{n}{12})$.
\begin{align*}
&P(|X|\le 10)\\
&=P(-10\le X \le 10)\\
&=P(-\frac{10}{\sqrt{\frac{n}{12}}} \le \frac{X}{\sqrt{\frac{n}{12}}} \le \frac{10}{\sqrt{\frac{n}{12}}})
&=2\Phi(\frac{10}{\sqrt{\frac{n}{12}}})-1
&> 0.96
\end{align*}
即$\Phi(\frac{10}{\sqrt{\frac{n}{12}}})>0.98$.查表得
\begin{align*}
&\frac{10}{\sqrt{\frac{n}{12}}}\ge 2.06\\
&n\le 282.78\\
&n \le 282
\end{align*}
所以最多282个数相加使得误差总和的绝对值小于10的概率不小于0.96.
\section*{5.}
\[
EX_k=\int_{0}^{1}6x^2(1-x)=\frac{1}{2}
\]
\[
EX_k^2=\int_{0}^{1}6x^3(1-x)=\frac{3}{10}
\]
\[
DX_k=EX_k^2-(EX_k)^2=\frac{3}{10}-\frac{1}{4}=\frac{1}{20}
\]
\[
\because\frac{1}{n^2}D(\sum X_k)=\frac{1}{20n}\rightarrow 0 (n \rightarrow \infty)
\]
$\therefore$由马尔科夫大数定理,序列\{$X_n$\}服从大数定理.
\[
\therefore
\frac{1}{n}\sum_{k=1}^{n}X_k
\overset{P}{\rightarrow}\frac{1}{n}\sum_{k=1}^{n}EX_k=\frac{1}{2}
\]
\section*{7.}
\begin{align*}
E(lnX_i)&=\int_{0}^{1}(lnx)*1 dx\\
&=xlnx|_0^1-\int_{0}^{1}xdlnx\\
&=-1
\end{align*}
因为$E(lnX_i)$有限,从而\{$lnX_i$\}服从辛钦大数定律。\\
\begin{align*}
&lnZ_n=\frac{1}{n}\sum_{i=1}^{n}lnX_i\overset{P}{\to}-1\\
&\because f(x)=e^x\text{连续}\\
&\therefore Z_n\overset{P}{\to}\frac{1}{e}
\end{align*}

\section*{补充}
已知$g(.,.)$在$(a,b)$处连续,即有
\[
\forall \epsilon >0, \exists \delta >0,|x-a|<\delta,|y-b|<\delta,\text{有}|g(x,y)-g(a,b)|<\epsilon
\]
已知$X_n\overset{P}{\rightarrow}a$,$Y_n\overset{P}{\rightarrow}b$,所以有
\begin{align*}
&\forall \epsilon>0\\
&\lim_{n \to \infty} P(|X_n-a|<\epsilon)=1\\
&\lim_{n \to \infty} P(|Y_n-b|<\epsilon)=1\\
&\therefore\lim_{n \to \infty} P(|X_n-a|<\epsilon,|Y_n-b|<\epsilon)=1  ~~~~~(*)
\end{align*}
下面证明$g(X_n,Y_n)\overset{P}{\rightarrow}g(a,b)$.取(*)式中$\epsilon=\delta$,有
\[
P(|X_n-a|<\delta,|Y_n-b|<\delta)=1
\]
\begin{align*}
&\forall \epsilon >0\\
&1=\lim_{n \to \infty}P(|x-a|<\delta,|y-b|<\delta)\\
&\le
\lim_{n \to \infty} P(|g(X_n,Y_n)-g(a,b)|<\epsilon)\text{由连续的定义}\\
&=1
\end{align*}
\[
\therefore \forall \epsilon>0 \lim_{n \to \infty} P(|g(X_n,Y_n)-g(a,b)|<\epsilon)=1
\]
\[
\therefore g(X_n,Y_n)\overset{P}{\to}g(a,b)
\]
\end{document}