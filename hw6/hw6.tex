\documentclass[a4paper,twocolumn]{ctexart}
\usepackage{ctex}
\usepackage{amsmath}


\graphicspath{{./figures/}}
%首行缩进两字符 利用\indent \noindent进行控制
\usepackage{indentfirst}
\setlength{\parindent}{2em}

%算法包
\usepackage{caption}
\usepackage{algorithm}
\usepackage{algorithmic}

%页边距包
\usepackage{geometry}
\geometry {left=2.0cm ,right=2.0cm,top=2.5cm,bottom=2.5cm}

%枚举
\usepackage{enumerate}

%代码
\usepackage{listings}

%算法input output
\renewcommand{\algorithmicrequire}{\textbf{Input:}} % Use Input in the format of Algorithm
\renewcommand{\algorithmicensure}{\textbf{Output:}} % Use Output in the format of Algorithm
% for fig without caption: #1: width/size; #2: fig file
\newcommand{\fignocaption}[2]{
	\begin{figure}[htp]
		\centering
		\includegraphics[#1]{#2}
	\end{figure}
}
% for fig with caption: #1: width/size; #2: fig file; #3: fig caption
\newcommand{\fig}[3]{
	\begin{figure}[htp]
		\centering
		\includegraphics[#1]{#2}
		\caption[labelInTOC]{#3}
	\end{figure}
}
%数学符号
\usepackage{amssymb}
\title{Probability And Mathmatical Statistics\\Homework 6}
\author{151220131 谢旻晖}
\date{}
\begin{document}
\maketitle
\section*{1}
$p$均匀地分布在[0,1]之间,考虑两枚非均匀硬币1和2,他们正面朝上的概率分别为$p_1$和$p_2$.\\
$p<p_1$时,硬币1正面朝上,$p<p_2$时,硬币2正面朝上.
因此,硬币1朝上时,必然有硬币2朝上.
P(硬币1朝上的次数$\ge k$)<=P(硬币2朝上的次数=$\ge k$)
即
\[
\sum_{i=k}^n\binom{n}{i}p_1^{i}(1-p_1)^{n-i}\le\sum_{i=k}^n\binom{n}{i}p_2^{i}(1-p_2)^{n-i}
\]
\section*{2}
\noindent \textbf{a)}\\
反证法,若$(v_1,v_2)\in E,v_1,v_2\in S(\rho)$,则在$\rho$中,$v_2$既领先于$v_1$,又落后于$v_1$,矛盾.\\
\noindent \textbf{b)}\\
设指示器随机变量$I_i$指示第$i$个点是否在独立集中,在为1,不在为0。$P(I_i=1)=\frac{1}{d_i+1}$,因为:考虑点$i$和他的邻居共$1+d_i$个点,当且仅当$i$在置换中的相对位置为$1+d_i$个点的最前面时,$i$才会在独立集中。
\begin{align*}
E(\sum_{i=1}^{n}I_i)&=\sum_{i=1}^{|V|}E(I_i)\\
&=\sum_{i=1}^{n}P(I_i=1)\\
&=\sum_{i=1}^{n}\frac{1}{d_i+1}
\end{align*}
\noindent \textbf{c)}\\
由握手定理$\sum_{i=1}^{n}d_i=nk$
\begin{align*}
E(\sum_{i=1}^{n}I_i)&=\sum_{i=1}^{n}\frac{1}{d_i+1}\\
&\ge \frac{n^2}{\sum_{i=1}^{n}(d_i+1)}\\
&=\frac{n^2}{nk+n}\\
&=\frac{n}{k+1}
\end{align*}
$E(\sum_{i=1}^{n}I_i)\ge \frac{n}{k+1}$,由期望的性质,G必定有不小于$\frac{n}{k+1}$的独立集.


\section*{3}
\noindent 初始化$A=\emptyset$\\
采样,以$p>0$的概率取点,加入集合$A$中.\\
修正,对于$V\ A$中那些点,若没有邻居在$A$中,就把他加入$A$中.这样就构成了一个支配集.\\
下面分析$E[A]$.不妨设随机变量$X$为采样后的点数,$Y$为修正时加入的点数.
\begin{align*}
E[A]&=E[X+Y]\\
&=E[X]+E[Y]\\
&=np+E[Y_1+Y_2+\ldots+Y_{n}]
\end{align*}
其中$Y_i$为指标随机变量,定义如下.易得$P(Y_i=1)=(1-p)^{d+1}$
\[Y_i=
\begin{cases}
1 &\text{没被采样到,且需要被修正(加入集合)}\\
0 & \text{else}
\end{cases}
\]
\begin{align*}
E[A]&=np+\sum_{i=1}^{n}E[Y_i]\\
&=np+\sum_{i=1}^{n}P(Y_i=1)\\
&=np+\sum_{i=1}^{n}(1-p)^{d+1}\\
&=np+n(1-p)^{d+1}\\
&\le np+e^{-p(d+1)}
\end{align*}
取$p=\frac{ln(d+1)}{d+1}$,代入
\[
E[A]\le \frac{n(1+ln(d+1))}{d+1}
\]
由期望论证,一定存在大小不超过$\frac{n(1+ln(d+1))}{d+1}$的支配集.
\section*{4}
在$K_n$中不断取与G同构的图进行边染色,一共染$k$轮,第i轮使用第i种颜色,若染色时边已有颜色,则覆盖掉原来的颜色.由于G中不含H子图,所以每轮染色中都不存在同色的H子图。下面只要证明$k$轮染色之后存在某种染色方案使得每条边都有颜色即可完成$E\to\ [k]$。\\
令$X_i$为指示器随机变量指示第$i$条边最后有没有被染色,有为1,没有为0.
\begin{align*}
&P(\text{存在边没有颜色})\\
&=P\left(X_1=0 \cup X_2=0 \ldots \cup X_{\binom{n}{2}}=0\right)\\
&\le \sum_{i=1}^{\binom{n}{2}}P(X_i=0)\\
&=\binom{n}{2} \left(1-\frac{m}{\binom{n}{2}}\right)^k\\
&=\frac{n(n-1)}{2}\left(1-\frac{2m}{n(n-1)}\right)^{\frac{n^2lnn}{m}}\\
&\le \frac{n(n-1)}{2}e^{-\frac{2nlnn}{n-1}}\\
&\le \frac{n^2}{2}\left(\frac{1}{n}\right)^{\frac{2n}{n-1}}\\
&<\frac{n^2}{2}\frac{1}{n}^n\\
&=\frac{1}{2}<1
\end{align*}
因此, $P(\text{边全都有颜色})>0$,得证。

\section*{5}
$X_i$指示第$i$个点是否为孤点,$X=\sum X_i$.使用条件期望不等式
\begin{align*}
P(\text{存在孤点})&=P(X>0)\\
&\ge \sum_{i=1}^{n}\frac{P(X_i=1)}{E[X|X_i=1]}\\
&=\sum_{i=1}^{n}\frac{(1-p)^{n-1}}{\sum_{j=1}^{n}P(X_j=1|X_i=1)}\\
&=\sum_{i=1}^{n}\frac{(1-p)^{n-1}}{1+(n-1)(1-p)^{n-2}}\\
&=\frac{n(1-p)^{n-1}}{1+(n-1)(1-p)^{n-2}}
\end{align*}
\begin{align*}
&\lim_{n \to \infty}\frac{n(1-p)^{n-1}}{1+(n-1)(1-p)^{n-2}}\\
&=\lim_{n \to \infty} \frac{(1-p)^{n-1}}{\frac{1}{n}+\frac{n-1}{n}(1-p)^{n-2}}\\
&=\lim_{n \to \infty} \frac{(1-p)^{n-1}}{(1-p)^{n-2}}\\
&=\lim_{n \to \infty} 1-p\\
&=1
\end{align*}
\[
P(\text{存在孤点}) \ge 1^{-} ,\text{when  }n\to \infty
\]
所以该图存在孤点的概率至少为$1-\epsilon$.
\section*{6}
\noindent \textbf{a)}\\
定义指示器随机变量$X_i$为第$i$组三个点是否构成三角形,一共有$\binom{n}{3}$组
\begin{align*}
E(X)&=E(\sum_{i=1}^{\binom{n}{3}}X_i)\\
&=\sum_{i=1}^{\binom{n}{3}}E(X_i)\\
&=\sum_{i=1}^{\binom{n}{3}}P(X_i)\\
&=\sum_{i=1}^{\binom{n}{3}}p^3\\
&=\frac{n(n-1)(n-2)}{3*2*n^3}\\
&<\frac{1}{6}
\end{align*}
由马尔科夫不等式,$X$非负:
\[
P(X\ge 1)\le \frac{E(X)}{1}\le \frac{1}{6}
\]
\noindent \textbf{b)}\\
利用条件期望不等式.
\begin{align*}
P(X\ge 1)&\ge \sum_{i=1}^{\binom{n}{3}}\frac{P(X_i=1)}{E[X|X_i=1]}\\
&=\sum_{i=1}^{\binom{n}{3}}\frac{P(X_i=1)}{\sum_{j=1}^{\binom{n}{3}}E[X_j|X_i=1]}\\
&=\sum_{i=1}^{\binom{n}{3}}\frac{P(X_i=1)}{\sum_{j=1}^{\binom{n}{3}}P(X_j=1|X_i=1)}
\end{align*}
下面计算$\sum_{j=1}^{\binom{n}{3}}P(X_j=1|X_i=1)$.考虑$j$和$i$的公共边条数分别为0,1,3时候对应的计数.
\[
\text{\# of common edges}
\begin{cases}
0& \begin{cases}
\binom{n-3}{3}  &\text{\# of common vertex =0}\\
3\binom{n-3}{2} &\text{\# of common vertex =1}
\end{cases}\\
1& 3\binom{n-3}{1}\\
3& 1
\end{cases}
\]
\[
\sum_{j=1}^{\binom{n}{3}}P(X_j=1|X_i=1)=p^3\left(\binom{n-3}{3}+3\binom{n-3}{2}\right)+p^23\binom{n-3}{1}+1
\]
\begin{align*}
P(X\ge 1)\ge \sum_{i=1}^{\binom{n}{3}} \frac{p^3}{p^3\left(\binom{n-3}{3}+3\binom{n-3}{2}\right)+p^23\binom{n-3}{1}+1}
\end{align*}
\begin{align*}
&\lim_{n \to \infty}P(X\ge 1)\\
&\ge \lim_{n \to \infty} \frac{n(n-1)(n-2)}{6n^3}
\frac{1}
{
\frac{\frac{(n-3)(n-4)(n-5)}{6}
+\frac{3(n-3)(n-4)}{2}
}{n^3}
+\frac{3(n-3)}{n^2}
+1}\\
&=\frac{1}{6}\frac{1}{\frac{7}{6}}\\
&=\frac{1}{7}
\end{align*}
\end{document}