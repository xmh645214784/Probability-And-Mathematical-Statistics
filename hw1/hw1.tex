\documentclass[a4paper]{ctexart}
\usepackage{ctex}
\usepackage{amsmath}


\graphicspath{{./figures/}}
%首行缩进两字符 利用\indent \noindent进行控制
\usepackage{indentfirst}
\setlength{\parindent}{2em}

%算法包
\usepackage{caption}
\usepackage{algorithm}
\usepackage{algorithmic}

%页边距包
\usepackage{geometry}
\geometry {left=2.0cm ,right=2.0cm,top=2.5cm,bottom=2.5cm}

%枚举
\usepackage{enumerate}

%代码
\usepackage{listings}

%算法input output
\renewcommand{\algorithmicrequire}{\textbf{Input:}} % Use Input in the format of Algorithm
\renewcommand{\algorithmicensure}{\textbf{Output:}} % Use Output in the format of Algorithm
% for fig without caption: #1: width/size; #2: fig file
\newcommand{\fignocaption}[2]{
	\begin{figure}[htp]
		\centering
		\includegraphics[#1]{#2}
	\end{figure}
}
% for fig with caption: #1: width/size; #2: fig file; #3: fig caption
\newcommand{\fig}[3]{
	\begin{figure}[htp]
		\centering
		\includegraphics[#1]{#2}
		\caption[labelInTOC]{#3}
	\end{figure}
}
%数学符号
\usepackage{amssymb}
\title{Probability And Mathmatical Statistics\\Homework 1}
\author{151220131 谢旻晖}
\date{}
\begin{document}
\maketitle
\section{习题一/1}
某人向目标射击3次,设第$i$次命中的事件为$A_i$,用$A_i$的运算表示下列事件:(1)只有第一次命中,(2)目标被命中,(3)至多命中一次,(4)至多命中两次,(5)至少命中两次

\noindent 解:\\
1.$A_{1}\bar{A_{2}}\bar{A_{3}}$\\
2.$\cup_{i=1}^{3}A_{i}$\\
3.$\bar{A_{2}}\bar{A_{3}}\cup\bar{A_{1}}\bar{A_{3}}\cup\bar{A_{1}}\bar{A_{2}}$\\
4.$\overline{A_{1}A_{2}A_{3}}$\\
5.$A_{2}A_{3}\cup A_{1}A_{3}\cup A_{1}A_{2}$

\section{习题一/4}
抛2枚骰子,以所抛数$m$,$n$为$A$的坐标,求点A(m,n)落入圆$x^2+y^2=19$内的概率。


\noindent 解:\\
\indent 样本点总数为所有的点数组合为$6*6=36$种。\\
\indent 设事件$X$为点$A(m,n)$落入上述圆内,则$X$涵盖的事件样本点有11个。\\
\[
Pr(X)=\frac{|X|}{|\Omega|}=\frac{11}{36}
\]

\section{习题一/6}
在50只柳丁中有3只强度太弱,如果这3只柳丁装在同一部件上,则这个部件强度就不合格,现有10个部件,每个部件装3个柳丁,若从50只柳丁中随机取用,问恰有1个部件强度不合格的概率是打算?


\noindent 解:\\
\indent 我们考虑柳丁是互相不同的,部件也是互相不同的。但装在同一部件上的三个柳丁不区别顺序。我们将柳丁标上号为1-50,三个坏的标为1-3,将10个部件表示为一个长度为10的list,list中的每个元素类型是一个三元集合。现在我们就将问题建模为
\[[\{X,X,X\},\{X,X,X\},....\{X,X,X\}]\]
用1-50往上述list中的三元集中不重复的填充,恰有一个集合是$\{1,2,3\}$的概率是多少。\\
\indent 样本点总数为从50个数中找30个数的排列依次填入,并消去三元集合内部的重复计数,即$\frac{A_{50}^{30}}{(3!)^{10}}$
\indent 恰有1个部件不合格为恰有一个集合为$\{1,2,3\}$的样本点个数为,首先确定一个集合是$\{1,2,3\}$,$C_{10}^1$,然后再用剩下来的47个数填入剩下的集合,有$\frac{A_{47}^{27}}{(3!)^9}$,根据乘法原则总共有$C_{10}^1 \frac{A_{47}^{27}}{(3!)^9}$.\\
\indent 由古典概型,

\begin{align*}
&Pr(\text{恰有1个部件强度不合格})\\
&=\frac{C_{10}^1 \frac{A_{47}^{27}}{(3!)^9}}{\frac{A_{50}^{30}}{(3!)^{10}}}\\
&=\frac{C_{10}^{1}*3!*A_{47}^{27}}{A_{50}^{30}}\\
&=\frac{C_{10}^1*3!*A_{50}^{30}}{50*49*48*A_{50}^{30}}\\
&=\frac{10}{C_{50}^3}
\end{align*}

\section{习题一/12}
平面上点$(p,q)$在$|p|\le1$,$|q|\le1$内等可能出现,求$x^2+px+q=0$有实根的概率。\\

\noindent 解:\\
样本空间$\Omega=\{(p,q)||p|\le1,|q|\le1\}$,设A为方程$x^2+px+q=0$有实根。则\\
\[
\Delta=p^2-4q\ge0
\]
\[
q\le p^2/4
\]
A为图中的阴影区域,由几何概型\\

\fignocaption{width=0.3\textwidth}{12.jpg}

\begin{align*}
&Pr(A)\\
&=\frac{2+\int_{-1}^{1}\frac{p^2}{4}dp}{2*2}\\
&=\frac{13}{24}
\end{align*}

\section{习题一/13}
将线段$(0,2a)$任意折成3折,求此3折线能构成三角形的概率。\\
\noindent 解:\\
设三折的长度分别为$x$,$y$,$2a-x-y$,那么我们有约束$x,y>0$,$x+y<2a$.
s
则样本空间为$\Omega=\{(x,y)|x,y>0$,$x+y<2a\}$.

其中能围成三角形的需要满足任意两边之和大于第三边:
\[
\begin{cases}
&x+y>2a-x-y\\
&x+2a-x-y>y\\
&y+2a-x-y>x
\end{cases}
\]
即$\{(x,y)|x+y>a,y<a,x<a\}$.那么由古典概型,
\fignocaption{width=0.3\textwidth}{13.jpg}
\[
Pr(\text{3折线能构成三角形)}=\frac{1}{4}
\]

\section{补充1}
考虑一种心形线$x^2+(y-\sqrt[3]{x^2})^2=1$(见图),请通过蒙特卡洛法求围成的面积。
\begin{lstlisting}{language=Matlab}
clear all
close all
% experiment result
count=0;            %# of samples drop inside curve
nr_iteration=10000000;%# of iterations
for i = 1:nr_iteration
	temp=rand(1,2)*4-2;
	x=temp(1);y=temp(2);
	if(y>=x^(2/3)-sqrt(1-x^2)&& y<=x^(2/3)+sqrt(1-x^2))
		count=count+1;
	end
end
disp(['The experiment result is ',num2str(count/nr_iteration*16)]);
\end{lstlisting}
代码运行结果:The experiment result is 3.1422.

\section{补充2}
考虑抛一枚均匀的硬币$n$次,给定一正整数$k$,考虑事件$A$,出现$\log_2 {n}+k$个连续正面朝上(假定$\log_2 n$为整数),证明
\[
P(A)\le 2^{-k}
\]
设$A_i$为事件第$i$次硬币起连续$\log_2 {n}+k$正面朝上的概率。

显然有
\[Pr(A_i)=\frac{1}{2^{\log_2 {n}+k}}=\frac{1}{n2^{k}}\]

然后
\[
Pr(A)=Pr(\bigcup_{i=1}^{n}Pr(A_i))\le \sum_{i=1}^{n} \frac{1}{n2^{k}}=\frac{1}{2^k}
\]

\section{补充3}
(非传递的骰子)考虑三枚均匀的骰子$A,B,C$,随机抛这三枚骰子,记它们的点数为$X,Y,Z$
\begin{itemize}
	\item 假设骰子各面的点数分别为$A:1,1,5,5,5,5$,$B:3,3,4,4,4,6$和$C:2,2,3,3,6,6$,证明$P(X>Y)=P(Y>Z)=P(Z>X)=\frac{5}{9}$.
	\item 设计三枚骰子(点数不超过6),使得$P(X>Y),P(Y>Z),P(Z>X)$均大于$\frac{5}{9}$.
\end{itemize}

\noindent 解:\\
\noindent(1)\\
\indent 样本空间$\Omega=\{\text{A,B,C结果三元组}\}$,认为相同数字但在不同面是不同的,共6*6*6个样本点。\\
很容易得到
\[
P(X>Y)=P(Y>Z)=P(Z>X)=\frac{4*5*6}{6*6*6}=\frac{5}{9}
\]

\noindent(2)\\
A: 1 4 4 4 4 4\\
B: 3 3 3 3 3 6\\
C: 2 2 2 5 5 5
\end{document}