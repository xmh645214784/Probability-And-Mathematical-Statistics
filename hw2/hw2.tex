\documentclass[a4paper,twocolumn]{ctexart}
\usepackage{ctex}
\usepackage{amsmath}


\graphicspath{{./figures/}}
%首行缩进两字符 利用\indent \noindent进行控制
\usepackage{indentfirst}
\setlength{\parindent}{2em}

%算法包
\usepackage{caption}
\usepackage{algorithm}
\usepackage{algorithmic}

%页边距包
\usepackage{geometry}
\geometry {left=2.0cm ,right=2.0cm,top=2.5cm,bottom=2.5cm}

%枚举
\usepackage{enumerate}

%代码
\usepackage{listings}

%算法input output
\renewcommand{\algorithmicrequire}{\textbf{Input:}} % Use Input in the format of Algorithm
\renewcommand{\algorithmicensure}{\textbf{Output:}} % Use Output in the format of Algorithm
% for fig without caption: #1: width/size; #2: fig file
\newcommand{\fignocaption}[2]{
	\begin{figure}[htp]
		\centering
		\includegraphics[#1]{#2}
	\end{figure}
}
% for fig with caption: #1: width/size; #2: fig file; #3: fig caption
\newcommand{\fig}[3]{
	\begin{figure}[htp]
		\centering
		\includegraphics[#1]{#2}
		\caption[labelInTOC]{#3}
	\end{figure}
}
%数学符号
\usepackage{amssymb}
\title{Probability And Mathmatical Statistics\\Homework 2}
\author{151220131 谢旻晖}
\date{}
\begin{document}
\maketitle
\section*{习题一 15}
\noindent
由条件概率公式,\\
\begin{align*}
&\because \frac{P(AB)}{P(A)}=P(B|A)\\
&\therefore P(AB)=\frac{1}{12}\\
&\because \frac{P(AB)}{P(B)}=P(A|B)\\
&\therefore P(B)=\frac{1}{6}\\
&P(\overline{A}\ \overline{B})\\
&=1-P(A\cup B)\\
&=1-(P(A)+P(B)-P(AB))\\
&=\frac{2}{3}
\end{align*}

\section*{习题一 16}
\noindent 设事件A为从中取2件至少有一件为次品,事件B为两件都是次品,则本题需要求$P(B|A)$.\\
\noindent 易得$P(A)=\frac{C_{4}^2+C_{6}^1*C_4^1}{C_{10}^{2}}$,$P(B)=\frac{C_{4}^2}{C_{10}^{2}}$.\\
\noindent 由条件概率公式
\[
P(B|A)=\frac{P(AB)}{P(A)}
\]
又因为$B\subseteq A$,所以$P(AB)=P(B)$,上式可化为
\begin{align*}
P(B|A)&=\frac{P(B)}{P(A)}\\
&=\frac{C_4^2}{C_4^1*C_6^1+C_4^2}\\
&=\frac{1}{5}
\end{align*}

\section*{习题一 20}
设事件$A_i (0\le i\le 2)$为一盒中分别有$i$只次品,由$\sum_{i=0}^{2}P(A_i)=1$,且$A_i$之间互斥,所以$A_0$,$A_1$,$A_2$构成一个完备事件组。\\
设事件B为该盒可以出厂。\\
(1)\\
由全概率公式
\begin{align*}
&P(B)\\
&=\sum_{i=0}^{2} P(B|A_i)P(A_i)\\
&=0.8+\frac{C_{19}^4}{C_{20}^4}*0.1+\frac{C_{18}^4}{C_{20}^4}*0.1\\
&=0.943
\end{align*}
(2)\\
题意欲求$P(A_0|B)$.由条件概率公式.\\
\begin{align*}
&P(A_0|B)=\frac{P(A_0B)}{P(B)}\\
&\because A_0 \subseteq B\\
&\therefore P(A_0B)=P(A_0)\\
&\therefore P(A_0|B)=\frac{P(A_0)}{P(B)}=0.848
\end{align*}

\section*{补充1}
\noindent 随机变量X为抛10次硬币正面朝上的次数,$X~B(10,\frac{1}{2})$.\\
\noindent(1)\\
\[
P(X=5)=C_{10}^5 (\frac{1}{2})^{5}(\frac{1}{2})^{5}=0.246
\]

\noindent(2)\\
\begin{align*}
P&=\sum_{i=6}^{10}P(X=i)\\
&=\frac{C_{10}^6+C_{10}^7+C_{10}^8+C_{10}^9+C_{10}^{10}}{2^{10}}\\
&=0.377
\end{align*}

\noindent(3)\\
样本空间为抛硬币10次出现的结果序列全集,事件A为所有$i=1,...5$,第$i$次抛硬币的结果和$11-i$次的结果相同。
$|\Omega|=2^{10}$,$|A|=2^5$
\[
P(A)=\frac{|A|}{\Omega}=\frac{1}{2^5}=\frac{1}{32}
\]

\noindent(4)\\
分类讨论,我们用0代表反面向上,用1代表正面向上,那么10次抛硬币的结果可以表示为一个10bit的序列。\\
(1)从第一个bit开始的连续4个及以上的正面朝上:有$2^6=64$种.\\
(2)非第一个bit开始的连续4个及以上的正面朝上:将[0 1 1 1 1]捆绑起来,它一共有6种放法,其余位置可以随意填入0与1,有$6*2^5=192$.其中[0 1 1 1 1 0 1 1 1 1]被重复计算。$192-1=191$\\
(3)1和2中其实有重复计数,[1 1 1 1 ? [0 1 1 1 1]]和[1 1 1 1 [0 1 1 1 1] ?],4种情况被计算了两次。\\
综上共有251种。\\
\[P=\frac{251}{2^{10}}=0.245\]
\section*{补充2}
采用数学归纳法,对$n$进行归纳。\\
\indent \textbf{$n=3$时},trivial.\\
\indent \textbf{假设$n=k$时},盒子中存在$n$个球时停止操作,白球数目等可能的为$1$到$n-1$中的某个数。\\
\indent \textbf{当$n=k+1$时},在盒中有$n-1$个球时,考虑完备事件组\{$A_i,1\le i \le n-2$\},其中$A_i$代表在盒中有$n-1$个球时白球数量为$i$.由假设,他们是等可能的,即$P(A_i)=\frac{1}{n-2},1\le i \le n-2$.\\
\indent 设事件$B_i,1\le i \le n-1$代表在盒中有$n$个球时白球数量为$i$.\\
那么,对于任意的$i\in [1,n-1],i \in Z$
\begin{align*}
P(B_i)&=\sum_{j=1}^{n-2}P(A_j)P(B_i|A_j)\\
&=P(A_i)P(B_i|A_i)+P(A_{i-1})P(B_i|A_{i-1})\\
&=\frac{1}{n-2}\frac{n-1-i}{n-1}+\frac{1}{n-2}\frac{i-1}{n-1}\\
&=\frac{1}{n-1}
\end{align*}
即$P(B_1)=P(B_2)=......=P(B_{n-1})=\frac{1}{n-1}$.假设对$n=k+1$仍然成立。\\
\indent 综上,证毕。



\section*{补充3}
\indent 应用推迟决定原则.\\
\indent 设事件$A_i$为$0\le i \le 5$为前9次投扔骰子的点数和模6为$i$.显然这是一个完备事件组。事件B为10次投扔骰子的和为6的倍数。\\
\indent 无论前9次点数和如何,最终10个骰子点数和取决于最后一个骰子的点数,且6种结果中只会有一种使得和为6的倍数.
即$P(B|A_i)=\frac{1}{6}$.\\
应用全概率公式:
\begin{align*}
P(B)&=\sum_{i=0}^{5}P(B|A_i)*P(A_i)\\
&=\sum_{i=0}^{5}\frac{1}{6}P(A_i)\\
&=\frac{1}{6}
\end{align*}
\section*{补充4}
定义$A,B,C$分别表示事件A,B,C会被释放,
$P(A)=P(B)=P(C)=\frac{1}{3}$,定义$W$表示事件牢头说$B$会被处决.
由全概率公式,
\begin{align*}
&P(W)\\
&=P(A)P(W|A)+P(B)P(W|B)+P(C)P(W|C)\\
&=\frac{1}{3}*p+\frac{1}{3}*0+\frac{1}{3}*1\\
&=\frac{p+1}{3}
\end{align*}

\end{document}