\documentclass[a4paper]{ctexart}
\usepackage{ctex}
\usepackage{amsmath}


\graphicspath{{./figures/}}
%首行缩进两字符 利用\indent \noindent进行控制
\usepackage{indentfirst}
\setlength{\parindent}{2em}

%算法包
\usepackage{caption}
\usepackage{algorithm}
\usepackage{algorithmic}

%页边距包
\usepackage{geometry}
\geometry {left=2.0cm ,right=2.0cm,top=2.5cm,bottom=2.5cm}

%枚举
\usepackage{enumerate}

%代码
\usepackage{listings}

%算法input output
\renewcommand{\algorithmicrequire}{\textbf{Input:}} % Use Input in the format of Algorithm
\renewcommand{\algorithmicensure}{\textbf{Output:}} % Use Output in the format of Algorithm
% for fig without caption: #1: width/size; #2: fig file
\newcommand{\fignocaption}[2]{
	\begin{figure}[htp]
		\centering
		\includegraphics[#1]{#2}
	\end{figure}
}
% for fig with caption: #1: width/size; #2: fig file; #3: fig caption
\newcommand{\fig}[3]{
	\begin{figure}[htp]
		\centering
		\includegraphics[#1]{#2}
		\caption[labelInTOC]{#3}
	\end{figure}
}
%数学符号
\usepackage{amssymb}
\title{Probability And Mathmatical Statistics\\Homework 2}
\author{151220131 谢旻晖}
\date{}
\begin{document}
\maketitle
\section*{习题一 15}
\noindent
由条件概率公式,\\
\begin{align*}
&\because \frac{P(AB)}{P(A)}=P(B|A)\\
&\therefore P(AB)=\frac{1}{12}\\
&\because \frac{P(AB)}{P(B)}=P(A|B)\\
&\therefore P(B)=\frac{1}{6}\\
&P(\overline{A}\ \overline{B})\\
&=1-P(A\cup B)\\
&=1-(P(A)+P(B)-P(AB))\\
&=\frac{2}{3}
\end{align*}

\section*{习题一 16}
\noindent 设事件A为从中取2件至少有一件为次品,事件B为两件都是次品,则本题需要求$P(B|A)$.\\
\noindent 易得$P(A)=\frac{C_{4}^2+C_{6}^1*C_4^1}{C_{10}^{2}}$,$P(B)=\frac{C_{4}^2}{C_{10}^{2}}$.\\
\noindent 由条件概率公式
\[
P(B|A)=\frac{P(AB)}{P(A)}
\]
又因为$B\subseteq A$,所以$P(AB)=P(B)$,上式可化为
\begin{align*}
P(B|A)&=\frac{P(B)}{P(A)}\\
&=\frac{C_4^2}{C_4^1*C_6^1+C_4^2}\\
&=\frac{1}{5}
\end{align*}

\section*{习题一 20}
设事件$A_i (0\le i\le 2)$为一盒中分别有$i$只次品,由$\sum_{i=0}^{2}P(A_i)=1$,且$A_i$之间互斥,所以$A_0$,$A_1$,$A_2$构成一个完备事件组。\\
设事件B为该盒可以出厂。\\
(1)\\
由全概率公式
\begin{align*}
&P(B)\\
&=\sum_{i=0}^{2} P(B|A_i)P(A_i)\\
&=0.8+\frac{C_{19}^4}{C_{20}^4}*0.1+\frac{C_{18}^4}{C_{20}^4}*0.1\\
&=0.943
\end{align*}
(2)\\
题意欲求$P(A_0|B)$.由条件概率公式.\\
\begin{align*}
&P(A_0|B)=\frac{P(A_0B)}{P(B)}\\
&\because A_0 \subseteq B\\
&\therefore P(A_0B)=P(A_0)\\
&\therefore P(A_0|B)=\frac{P(A_0)}{P(B)}=0.848
\end{align*}
\end{document}