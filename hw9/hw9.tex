\documentclass[a4paper,twocolumn]{ctexart}
\usepackage{ctex}
\usepackage{amsmath}


\graphicspath{{./figures/}}
%首行缩进两字符 利用\indent \noindent进行控制
\usepackage{indentfirst}
\setlength{\parindent}{2em}

%算法包
\usepackage{caption}
\usepackage{algorithm}
\usepackage{algorithmic}

%页边距包
\usepackage{geometry}
\geometry {left=2.0cm ,right=2.0cm,top=2.5cm,bottom=2.5cm}

%枚举
\usepackage{enumerate}

%代码
\usepackage{listings}

%算法input output
\renewcommand{\algorithmicrequire}{\textbf{Input:}} % Use Input in the format of Algorithm
\renewcommand{\algorithmicensure}{\textbf{Output:}} % Use Output in the format of Algorithm
% for fig without caption: #1: width/size; #2: fig file
\newcommand{\fignocaption}[2]{
	\begin{figure}[htp]
		\centering
		\includegraphics[#1]{#2}
	\end{figure}
}
% for fig with caption: #1: width/size; #2: fig file; #3: fig caption
\newcommand{\fig}[3]{
	\begin{figure}[htp]
		\centering
		\includegraphics[#1]{#2}
		\caption[labelInTOC]{#3}
	\end{figure}
}
%数学符号
\usepackage{amssymb}
\title{Probability And Mathmatical Statistics\\Homework 9}
\author{151220131 谢旻晖}
\date{}
\begin{document}
\maketitle
\section*{3.}
\noindent\textbf{(1)}\\
\begin{align*}
S^2&=\frac{\sum_{i=1}^{n}X_i^2}{n-1}\\
&\frac{\sum_{i=1}^{n}(X_i-\bar{X})^2}{n-1}\\
&=\frac{\sum_{i=1}^{n}(X_i^2-2X_i\bar{X}+\bar{X}^2)}{n-1}\\
&=\frac{\sum_{i=1}^{n}X_i^2-2\bar{X}\sum_{i=1}^{n}X_i+n\bar{X}^2}{n-1}\\
&=\frac{\sum_{i=1}^{n}X_i^2-2\bar{X}n\bar{X}+n\bar{X}^2}{n-1}\\
&=\frac{1}{n-1}\left[\sum_{i=1}^{n}X_i^2-n\bar{X}^2\right]
\end{align*}
\noindent\textbf{(2)}\\
\begin{align*}
E(S^2)&=E(\frac{1}{n-1}\left[\sum_{i=1}^{n}X_i^2-n\bar{X}^2\right])\\
&=\frac{1}{n-1}\left[\sum_{i=1}^{n}EX_i^2-nE\bar{X}^2\right]\\
&=\frac{1}{n-1}\left[\sum_{i=1}^{n}(DX_i+(EX_i)^2)-n\left(D\bar{X}+\left(E\bar{X}\right)^2\right)\right]\\
&=\frac{1}{n-1}\left[n(\sigma^2+\mu^2)-n\left(\frac{n\sigma^2}{n^2}+\mu^2\right)\right]\\
&=\frac{1}{n-1}\left[(n-1)\sigma^2\right]=\sigma^2
\end{align*}
\section*{8.}
\begin{align*}
&EY\\
&=\sum_{i=1}^{n}E\left(X_i+X_{n+i}-2\bar{X}\right)^2\\
&=\sum_{i=1}^{n}\left[
D\left(X_i+X_{n+i}-2\bar{X}\right)+\left(E\left(X_i+X_{n+i}-2\bar{X}\right)\right)^2
\right]\\
&=\sum_{i=1}^{n}\left[
\left(2\sigma^2+4D\bar{X}-4cov(\bar{X},X_i)-4cov(\bar{X},X_{n+i})\right)
+
0
\right]\\
&=\sum_{i=1}^{n}\left[
2\sigma^2+4\frac{\sigma^2}{2n}-4\frac{DX_i}{2n}-4\frac{DX_{n+i}}{2n}
\right]\\
&=\sum_{i=1}^{n}\left[2\sigma^2-\frac{2\sigma^2}{n}\right]=(2n-2)\sigma^2
\end{align*}
\section*{9.}
\begin{align*}
&\frac{X_{n+1}-\bar{X}}{S}\sqrt{\frac{n}{n+1}}\\
&=\frac{\frac{X_{n+1}-\bar{X}}{\sigma\sqrt{\frac{1+n}{n}}}}{\sqrt{\frac{\frac{(n-1)S^2}{\sigma^2}}{n-1}}}
\end{align*}
\begin{align*}
\text{令}A&=\frac{X_{n+1}-\bar{X}}{\sigma\sqrt{\frac{1+n}{n}}}\\
B&=\frac{(n-1)S^2}{\sigma^2}\\
\text{则原式化为}&\frac{A}{\sqrt{\frac{B}{n-1}}}
\end{align*}
由定理$B~\sim~~\chi^2(n-1)$.A是独立的正态分布随机变量之和,仍然是正态分布,且$A~\sim~~N(0,1)$.\\
由定理$S^2$与$\bar{X}$独立,且显然有$S^2$与$X_{n+1}$独立.所以有
$A$与$B$独立。\\
由t分布定义,原式服从$t(n-1)$
\section*{10.}
\noindent\textbf{(1)}\\
\[
\bar{X}~\sim~~N(12,\frac{4}{5})
\]
\begin{align*}
&P(\bar{X}>13)\\
&=P(\frac{\bar{X}-12}{\sqrt{\frac{4}{5}}}>\frac{\sqrt{5}}{2})\\
&=1-\Phi(\frac{\sqrt{5}}{2})\\
&=1-0.8686=0.1314
\end{align*}
\noindent\textbf{(2)}\\
\begin{align*}
&P(\underset{1\le i\le 5}{\min}X_i<10)\\
&=1-P(\underset{1\le i\le 5}{\min}X_i\ge10)
&=1-\Pi_{i=1}^5P(X_i\ge 10)\\
&=1-\Pi_{i=1}^5 1-\Phi(-1)\\
&=1-\Pi_{i=1}^5 \Phi(1)\\
&=0.5785
\end{align*}
\noindent\textbf{(3)}\\
\begin{align*}
&P(\underset{1\le i\le 5}{\max}X_i>15)\\
&=1-P(\underset{1\le i\le 5}{\max}X_i\le 15)
&=1-\Pi_{i=1}^5P(X_i\le 15)\\
&=1-\Pi_{i=1}^5\Phi(1.5)\\
&=0.2923
\end{align*}
\section*{11.}
\textbf{从答案来看$S_1^2$和$S_2^2$应该为样本修正方差,所求的也为样本修正方差,题目有误}\\
设合并后的样本为$Z_i$,样本均值和样本修正方差为$\bar{Z}$和$S_Z^2$
\begin{align*}
\bar{Z}&=\frac{n_1\bar{X}+n_2\bar{Y}}{n_1+n_2}\\
S_Z^2&=\frac{\sum_{i=1}^{n_1+n_2}Z_i^2-(n_1+n_2)\bar{Z}^2}{n_1+n_2-1}\\
&=\frac{(n_1-1)S_1^2+n_1\bar{X}^2+(n_2-1)S_2^2+n_2\bar{Y}^2-(n_1+n_2)\bar{Z}^2}{n_1+n_2-1}\\
&=\frac{(n_1-1)S_1^2+n_1\bar{X}^2+(n_2-1)S_2^2+n_2\bar{Y}^2-\frac{\left(n_1\bar{X}+n_2\bar{Y}\right)^2}{n_1+n_2}}{n_1+n_2-1}
\end{align*}
\section*{补充}
\begin{align*}
&\frac{(X_1+X_2)^2}{(X_1-X_2)^2}\\
&=\frac{\frac{(X_1+X_2)^2}{2\sigma^2}}
{\frac{(X_1-X_2)^2}{2\sigma^2}}
\end{align*}
\begin{align*}
\because \frac{X_1+X_2}{\sqrt{2}\sigma}~~\sim~N(0,1)\\
\therefore \frac{(X_1+X_2)^2}{2\sigma^2}~~\sim~\chi^2(1)\\
\because \frac{X_1-X_2}{\sqrt{2}\sigma}~~\sim~N(0,1)\\
\therefore \frac{(X_1-X_2)^2}{2\sigma^2}~~\sim~\chi^2(1)
\end{align*}
下面证明$\frac{(X_1+X_2)^2}{2\sigma^2}$和$\frac{(X_1-X_2)^2}{2\sigma^2}$独立,只要证明$X_1+X_2$和$X_1-X_2$独立,又由于他们都是多元正态分布,独立与不相关等价,只要证明:$cov(X_1+X_2,X_1-X_2)=0$即可。
\begin{align*}
cov(X_1+X_2,X_1-X_2)&=D(X_1)-cov(X_1,X_2)+cov(X_1,X_2)-D(X_2)\\
&=\sigma^2-\sigma^2=0
\end{align*}
由F分布定义,原式子服从$F(1,1)$.
\end{document}