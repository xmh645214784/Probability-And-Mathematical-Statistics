\documentclass[a4paper,twocolumn]{ctexart}
\usepackage{ctex}
\usepackage{amsmath}


\graphicspath{{./figures/}}
%首行缩进两字符 利用\indent \noindent进行控制
\usepackage{indentfirst}
\setlength{\parindent}{2em}

%算法包
\usepackage{caption}
\usepackage{algorithm}
\usepackage{algorithmic}

%页边距包
\usepackage{geometry}
\geometry {left=2.0cm ,right=2.0cm,top=2.5cm,bottom=2.5cm}

%枚举
\usepackage{enumerate}

%代码
\usepackage{listings}

%算法input output
\renewcommand{\algorithmicrequire}{\textbf{Input:}} % Use Input in the format of Algorithm
\renewcommand{\algorithmicensure}{\textbf{Output:}} % Use Output in the format of Algorithm
% for fig without caption: #1: width/size; #2: fig file
\newcommand{\fignocaption}[2]{
	\begin{figure}[htp]
		\centering
		\includegraphics[#1]{#2}
	\end{figure}
}
% for fig with caption: #1: width/size; #2: fig file; #3: fig caption
\newcommand{\fig}[3]{
	\begin{figure}[htp]
		\centering
		\includegraphics[#1]{#2}
		\caption[labelInTOC]{#3}
	\end{figure}
}
%数学符号
\usepackage{amssymb}
\title{Probability And Mathmatical Statistics\\Homework 5}
\author{151220131 谢旻晖}
\date{}
\begin{document}
\maketitle

\section*{习题四4}
\noindent1.\\
\begin{center}
\begin{tabular}{cccc}
	\hline
	\hline
	(U,V)&(1,1)&(2,1)&(2,2)\\
	P&$\frac49$&$\frac49$&$\frac19$\\
	\hline
\end{tabular}
\end{center}
\noindent2.\\
\[
E[U]=\frac{14}{9}
\]
\[
E[V]=\frac{10}{9}
\]
\noindent3.\\
\begin{center}
	\begin{tabular}{cccc}
		\hline
		\hline
		UV&1&2&4\\
		P&$\frac{4}{9}$&$\frac{4}{9}$&$\frac{1}{9}$\\
		\hline
	\end{tabular}
\end{center}
\begin{align*}
&Cov(U,V)\\
&=E[UV]-E[U]E[V]\\
&=\frac{16}{9}-\frac{140}{81}\\
&=\frac{4}{81}
\end{align*}
\section*{习题四19}
两个泊松分布参数分别为1和2,由方差的公式
\begin{align*}
&E[X+Y]^2\\
&=Var[X+Y]+(E[X+Y])^2\\
&=Var[X]+Var[Y]+(E[X]+E[Y])^2\\
&=1+2+9\\
&=12
\end{align*}
\section*{习题四23}
\[
P(|X|\ge x)=P(f(|X|)\ge f(x))\le \frac{E[|f(X)|]}{f(x)}
\]
\section*{2}
令$X_i=
\begin{cases}
1 &\text{第i个数为不动点}\\
0 &\text{第i个数非不动点}
\end{cases}
$
令$X$为不动点的个数,则$X=\sum_{i=1}^{n}X_i$.\\
\begin{align*}
E[X]&=E[\sum_{i=1}^{n}X_i]\\
&=\sum_{i=1}^{n}E[X_i]\\
&=\sum_{i=1}^{n}P(X_i=1)\\
&=\sum_{i=1}^{n}\frac{1}{n}\\
&=1
\end{align*}
\begin{align*}
E[X^2]&=E[(\sum_{i=1}^{n}X_i)^2]\\
&=E[\sum_{i,j=1}^{n}X_iX_j]\\
&=\sum_{i,j=1}^{n}E[X_iX_j]\\
&=\sum_{i=1}^{n}E[X_i^2]+\sum_{i\neq j}E[X_iX_j]\\
&=1+\sum_{i\neq j}P(X_iX_j=1)\\
&=1+\sum_{i\neq j}P(X_i=1,X_j=1)\\
&=1+\sum_{i\neq j}\frac{1}{n(n-1)}\\
&=1+1=2
\end{align*}
\begin{align*}
Var(X)=E[X^2]-(E[X])^2=2-1=1
\end{align*}
\section*{3}
设随机变量$X_i$为第i天股票的价格.由全期望公式
\begin{align*}
E[X_i]&=pE[X_i|\text{第$i-1$天涨}]+qE[X_i|\text{第$i-1$天跌}]\\
&=pE[X_{i-1}*r]+qE[X_{i-1}*\frac{1}{r}]\\
&=(pr+\frac{q}{r})E[X_{i-1}] \text{~~~~~~~~~(期望的线性性质)}
\end{align*}
\[
\because E[X_1]=1
\therefore
E[X_d]=(pr+\frac{q}{r})^{d-1}
\]
\begin{align*}
E[X_i^2]&=pE[X_i^2|\text{第$i-1$天涨}]+qE[X_i^2|\text{第$i-1$天跌}]\\
&=pE[X_{i-1}^2*r^2]+qE[X_{i-1}^2*\frac{1}{r^2}]\\
&=(pr^2+\frac{q}{r^2})E[X_{i-1}^2]
\end{align*}
\[
\because E[X_1^2]=1
\therefore
E[X_d^2]=(pr^2+\frac{q}{r^2})^{d-1}
\]
\begin{align*}
Var[X_d]&=E[X_d^2]-(E[X_d])^2\\
&=(pr^2+\frac{q}{r^2})^{d-1}-(pr+\frac{q}{r})^{2d-2}
\end{align*}
where $q=1-p$
\section*{4}
\noindent \textbf{a)}
\begin{center}
	\begin{tabular}{ccc}
		\hline
		\hline
		$Y_i$&0&1\\
		$P$&$\frac{1}{2}$&$\frac{1}{2}$\\
		\hline
	\end{tabular}
\end{center}
\noindent\textbf{ b)}\\
取特定的$Y_1=Y_2...Y_{\binom{n}{2}}=1$.显然有
\[
P(Y_1=1)P(Y_2=1)...P(Y_{\binom{2}{1}}=1)=\frac{1}{2^{\binom{n}{2}}}
\]
而
\[
P(Y_1=1\wedge Y_2=1 ...\wedge Y_n=1)=0
\]
说明如下:考虑$n$个随机比特中某3个$a,b,c$,若$a\oplus b=1$且$a\oplus c=1$,那么$b\oplus c=b\oplus a \oplus a \oplus c=0$.
所以不存在所有异或生成的比特均为1的情况。\\
因此,$
P(Y_1=1)P(Y_2=1)...P(Y_{\binom{2}{1}}=1)\neq P(Y_1=1\wedge Y_2=1 ...\wedge Y_n=1)
$\\
$Y_i,i=1,2...\binom{n}{2}$并不相互独立.

\noindent \textbf{c)}
\begin{align*}
E[Y_iY_j]&=0*P(Y_iY_j=0)+1*P(Y_iY_j=1)\\
&=P(Y_iY_j=1)\\
&=P(Y_i=1\wedge Y_j=1)\\
&\overset{*}{=}P(Y_i=1)P(Y_j=1)\\
&=E[Y_i]E[Y_j]
\end{align*}
其中带*号的等号作进一步说明:分两种情况.\\
若$Y_i$和$Y_j$是由四个比特生成,则显然他们是独立的,有$P(Y_i=1\wedge Y_j=1)=P(Y_i=1)P(Y_j=1)$。\\
若$Y_i$和$Y_j$是由三个比特生成,即他们之间使用了公共的比特,不妨设$Y_i=a\oplus b$,$Y_j=b\oplus c$.样本空间为$(a,b,c)$三元组个数,共8个,其中$Y_i=1\wedge Y_j=1$有{(0,1,0),(1,0,1)}两种,$P(Y_i=1\wedge Y_j=1)=\frac{1}{4}$.而$Y_i=1$和$Y_j=1$各4种,$P(Y_i=1)=\frac{1}{2}$,$P(Y_j=1)=\frac{1}{2}$,有$P(Y_i=1\wedge Y_j=1)=P(Y_i=1)P(Y_j=1)$。

\noindent \textbf{d)}
\begin{align*}
Var[Y]=Var[\sum Y_i]&=\sum Var[Y_i]+2\sum_{1\le i<j\le \binom{n}{2}}Cov[Y_i,Y_j]\\
&=\sum Var[Y_i]\\
&=\sum E[Y_i^2]-(E[Y_i])^2\\
&=\sum \frac{1}{4}=\frac{\binom{n}{2}}{4}
\end{align*}

\noindent\textbf{e)}
使用切比雪夫不等式
\[
P(|Y-E(Y)|\ge n)\le \frac{Var[Y]}{n^2}=\frac{\binom{n}{2}}{4n^2}=\frac{n-1}{8n}
\]
\end{document}