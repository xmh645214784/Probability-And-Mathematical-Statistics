\documentclass[a4paper,twocolumn]{ctexart}
\usepackage{ctex}
\usepackage{amsmath}


\graphicspath{{./figures/}}
%首行缩进两字符 利用\indent \noindent进行控制
\usepackage{indentfirst}
\setlength{\parindent}{2em}

%算法包
\usepackage{caption}
\usepackage{algorithm}
\usepackage{algorithmic}

%页边距包
\usepackage{geometry}
\geometry {left=2.0cm ,right=2.0cm,top=2.5cm,bottom=2.5cm}

%枚举
\usepackage{enumerate}

%代码
\usepackage{listings}

%算法input output
\renewcommand{\algorithmicrequire}{\textbf{Input:}} % Use Input in the format of Algorithm
\renewcommand{\algorithmicensure}{\textbf{Output:}} % Use Output in the format of Algorithm
% for fig without caption: #1: width/size; #2: fig file
\newcommand{\fignocaption}[2]{
	\begin{figure}[htp]
		\centering
		\includegraphics[#1]{#2}
	\end{figure}
}
% for fig with caption: #1: width/size; #2: fig file; #3: fig caption
\newcommand{\fig}[3]{
	\begin{figure}[htp]
		\centering
		\includegraphics[#1]{#2}
		\caption[labelInTOC]{#3}
	\end{figure}
}
%数学符号
\usepackage{amssymb}
\title{Probability And Mathmatical Statistics\\Homework 4}
\author{151220131 谢旻晖}
\date{}
\begin{document}
\maketitle
\section*{1.}
\subsection*{a}
\begin{align*}
&P(X=Y)\\
&=\sum_{k=1}^{+\infty}P(X=Y=k)\\
&=\sum_{k=1}^{+\infty}P(X=k)P(Y=k)\\
&=\sum_{k=1}^{+\infty}\left(\left(1-p\right)\left(1-q\right)\right)^{k-1}pq\\
&=\frac{pq}{p+q-pq}
\end{align*}
\subsection*{b}
\begin{align*}
&P(\min\left(X,Y\right)=k)\\
&=P(X=k,Y\ge k)+P(X>k,Y=k)\\
&=P(X=k)P(Y\ge k)+P(X>k)P(Y=k)\\
&=(1-p)^{k-1}p(1-q)^{k-1}+(1-p)^k(1-q)^{k-1}q\\
&=(1-p)^{k-1}(1-q)^{k-1}(p+q-pq)
\end{align*}

\subsection*{c}
\noindent \textit{Solution1.}\\
由定义,先求概率$P(\max(X,Y))=k$.\\
\begin{align*}
&P(\max(X,Y)=k)\\
&=P(X=k,Y\le k)+P(X<k,Y=k)\\
&=P(X=k)P(Y\le k)+P(X<k)P(Y=k)\\
&=P(X=k)(1-P(Y>k))+(1-P(X\ge k))P(Y=k)\\
&=(1-p)^{k-1}p(1-(1-q)^k)+(1-q)^{k-1}q(1-(1-p)^{k-1})\\
&=(1-p)^{k-1}p+(1-q)^{k-1}q-(p+q-pq)(1-p)^{k-1}(1-q)^{k-1}
\end{align*}
再求期望:
\begin{align*}
&E(\max(X,Y))\\
&=\sum_{k=1}^{+\infty}kP(\max (X,Y)=k)\\
&=\sum_{k=1}^{+\infty}k(1-p)^{k-1}p+\sum_{k=1}^{+\infty}k(1-q)^{k-1}q-\sum_{k=1}^{+\infty}(p+q-pq)(1-p)^{k-1}(1-q)^{k-1}\\
&=\frac{1}{p}+\frac{1}{q}-\frac{1}{p+q-pq}
\end{align*}
\noindent \textit{Solution2.}\\
在b.的推导中发现$k=\min(X,Y)\sim G(p+q-pq)$,因此有$E[\min(X,Y)]=(p+q-pq)^{-1}$
\begin{align*}
E[\max(X,Y)]&=E[X]+E[Y]-E[\min(X,Y)]\\
&=\frac{1}{p}+\frac{1}{q}-\frac{1}{p+q-pq}
\end{align*}
\subsection*{d}
\begin{align*}
&E[X|X\le Y]\\
&=\sum_{x=0}^{+\infty}xP(X=x|X\le Y)\\
&=\sum_{x=0}^{+\infty}x\frac{P(X=x,X\le Y)}{P(X\le Y)}\\
\end{align*}
其中
\[P(X=x,X\le Y)=(1-p)^{x-1}p(1-q)^{x-1}\]
\begin{align*}
&P(X\le Y)\\
&=\sum_{k=0}^{+\infty}P(X=k,Y\ge k)\\
&=\sum (1-p)^{k-1}p(1-q)^{k-1}\\
&=p\sum \left((1-p)(1-q)\right)^{k-1}\\
&=\frac{p}{p+q-pq}
\end{align*}
\begin{align*}
\therefore\\
&E[X|X\le Y]\\
&=\sum_{x=0}^{+\infty}(p+q-pq)(1-p)^{x-1}(1-q)^{x-1}\\
&=\frac{1}{p+q-pq}
\end{align*}

\section*{2.}
定义随机变量X为抛到连续出现两个6所抛次数,定义两个指示器随机变量Y,Z为
\[
Y=
\begin{cases}
1&\text{第一次抛掷结果为6}\\
0&\text{第一次抛掷结果不为6}
\end{cases}
\]
\[
Z=
\begin{cases}
1&\text{第二次抛掷结果为6}\\
0&\text{第二次抛掷结果不为6}
\end{cases}
\]
\begin{align*}
&E[X]\\
&=P(Y=0)E[X|Y=0]+P(Y=1)E(X|Y=1)\\
&=\frac{5}{6}(1+E[X])+\frac{1}{6}E(X|Y=1)\\
&=\frac{5}{6}(1+E[X])+\frac{1}{6}\left(P(Z=0)E[X|Y=1,Z=0]+P(Z=1)E[X|Y=1,Z=1]\right)\\
&=\frac{5}{6}(1+E[X])+\frac{1}{6}\left(\frac{5}{6}(2+E[X])+\frac{1}{6}*2\right)
\end{align*}
解得
\[
E[X]=42
\]

\section*{3.}
连着抛掷两次,若结果为正反,则为1;若结果为反正,则为0;若结果为正正或反反,则重新再抛两次。

易见$P(generate\ 1)=P(generate\ 0)=p(1-p)$.将连续的两次绑定为整体,所需连续抛掷整体的次数服从参数为$2p(1-p)$的几何分布,$E=\frac{1}{2p(1-p)}$,再乘以2为$E=\frac{1}{p(1-p)}$
\end{document}