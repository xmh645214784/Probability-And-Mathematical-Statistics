\documentclass[a4paper,twocolumn]{ctexart}
\usepackage{ctex}
\usepackage{amsmath}


\graphicspath{{./figures/}}
%首行缩进两字符 利用\indent \noindent进行控制
\usepackage{indentfirst}
\setlength{\parindent}{2em}

%算法包
\usepackage{caption}
\usepackage{algorithm}
\usepackage{algorithmic}

%页边距包
\usepackage{geometry}
\geometry {left=2.0cm ,right=2.0cm,top=2.5cm,bottom=2.5cm}

%枚举
\usepackage{enumerate}

%代码
\usepackage{listings}

%算法input output
\renewcommand{\algorithmicrequire}{\textbf{Input:}} % Use Input in the format of Algorithm
\renewcommand{\algorithmicensure}{\textbf{Output:}} % Use Output in the format of Algorithm
% for fig without caption: #1: width/size; #2: fig file
\newcommand{\fignocaption}[2]{
	\begin{figure}[htp]
		\centering
		\includegraphics[#1]{#2}
	\end{figure}
}
% for fig with caption: #1: width/size; #2: fig file; #3: fig caption
\newcommand{\fig}[3]{
	\begin{figure}[htp]
		\centering
		\includegraphics[#1]{#2}
		\caption[labelInTOC]{#3}
	\end{figure}
}
%数学符号
\usepackage{amssymb}
\title{Probability And Mathmatical Statistics\\Homework 10}
\author{151220131 谢旻晖}
\date{}
\begin{document}
\maketitle
\section*{3}
\noindent\textbf{矩估计}\\
\begin{align*}
EX&=\int_{0}^{+\infty}\frac{1}{\sqrt{2\pi}\sigma}e^{-\frac{(lnx-\mu)^2}{2\sigma^2}}dx\\
\text{令}t&=\frac{lnx-\mu}{\sqrt{2}\sigma}\\
EX&=\int_{-\infty}^{+\infty}\frac{1}{\sqrt{2\pi}\sigma}e^{-t^2}\sqrt{2}\sigma e^{\sqrt{2}\sigma t+\mu}dt\\
&=\frac{1}{\sqrt{\pi}}\int_{-\infty}^{+\infty}e^{-t^2+\sqrt{2}\sigma t+\mu}dt\\
&=\frac{e^{\mu+\frac{\sigma^2}{2}}}{\sqrt{\pi}}\int_{-\infty}^{+\infty}e^{-(t-\frac{\sqrt{2}}{2}\sigma)^2}dt\\
\text{令}u&=t-\frac{\sqrt{2}}{2}\sigma\\
EX&=\frac{\sqrt{\pi}e^{\mu+\frac{\sigma^2}{2}}}{\sqrt{\pi}}\int_{-\infty}^{+\infty}\frac{1}{\sqrt{\pi}}e^{-u^2}du\\
&=e^{\mu+\frac{\sigma^2}{2}}~~~~~~~~~~(\text{右边为}N(0,\frac{1}{\sqrt{2}})\text{的概率积分})\\
EX^2&=\int_{0}^{+\infty}\frac{x}{\sqrt{2\pi}\sigma}e^{-\frac{(lnx-\mu)^2}{2\sigma^2}}dx\\
\text{令}t&=\frac{lnx-\mu}{\sqrt{2}\sigma}\\
&=\int_{-\infty}^{+\infty}\frac{1}{\sqrt{\pi}}e^{2\sqrt{2}\sigma t+2\mu -t^2}dt\\
&=\int_{-\infty}^{+\infty}\frac{1}{\sqrt{\pi}}e^{-(t-\sqrt{2}\sigma)^2}e^{2\sigma^2+2\mu}dt\\
&=\frac{e^{2\sigma^2+2\mu}}{\sqrt{\pi}}\int_{-\infty}^{+\infty}e^{-(t-\sqrt{2}\sigma)^2}dt\\
&=e^{2\sigma^2+2\mu}
\end{align*}
使用样本矩代替标准矩,得
\[
\begin{cases}
	\bar{X}&=e^{\mu+\frac{\sigma^2}{2}}\\
	\frac{1}{n}\sum_{i=1}^{n} X_i^2&=e^{2\mu+2\sigma^2}
\end{cases}
\]
解得
\begin{align*}
\hat{\mu}&=\frac{4ln\bar{X}-ln\frac{\sum_{i=1}^{n}X_i^2}{n}}{2}\\
\hat{\sigma^2}&=ln\frac{\sum_{i=1}^{n}X_i^2}{n}-2ln\bar{X}
\end{align*}
\noindent\textbf{极大似然估计}\\
似然函数为
\begin{align*}
L(\mu,\sigma^2)&=\Pi_{i=1}^{n}(2\pi \sigma^2)^{-\frac{1}{2}}X_i^{-1}e^{-\frac{(lnX_i-\mu)^2}{2\sigma^2}}\\
&=(2\pi\sigma^2)^{-\frac{n}{2}}\frac{1}{\Pi_{i=1}^{n}X_i}e^{-\frac{\sum_{i=1}^{n}(lnX_i-\mu)^2}{2\sigma^2}}\\
lnL(\mu,\sigma^2)&=-\frac{n}{2}ln(2\pi\sigma^2)+ln\frac{1}{\Pi_{i=1}^{n}X_i}\\
&-\frac{\sum_{i=1}^{n}(lnX_i-\mu)^2}{2\sigma^2}
\end{align*}
\[
\begin{cases}
\frac{\partial lnL}{\partial \sigma^2}&=-\frac{n}{2}\frac{2\pi}{2\pi\sigma^2}+\frac{\sum_{i=1}^{n}(lnX_i-\mu)^2}{2(\sigma^2)^2}=0\\
\frac{\partial lnL}{\partial \mu}&=-\frac{\sum_{i=1}^{n}2(\mu-lnX_i)}{2\sigma^2}=0
\end{cases}
\]
解得
\begin{align*}
\hat{\mu}&=\frac{\sum_{i=1}^{n}lnX_i}{n}\\
\hat{\sigma^2}&=\frac{\sum_{i=1}^{n}\left[lnX_i-\sum_{j=1}^{n}\frac{lnX_j}{n}\right]}{n}
\end{align*}
\section*{4}
\noindent\textbf{矩估计}\\
令$Y=X-\mu$.\\
\[
p(y;\mu,\theta)=\begin{cases}
\frac{1}{\theta}e^{-\frac{y}{\theta}},y\ge0\\
0~~~~~~~~O.W.
\end{cases}
\]
则$Y~\sim~E(\theta)$,$EY=\theta$,$EY^2=2\theta^2$.
\begin{align*}
EX&=E(Y+\mu)=\theta+\mu\\
EX^2&=E(Y+\mu)^2=2\theta^2+2\mu\theta+\mu^2\\
\therefore\\
\hat{\mu}&=\bar{X}-S_n\\
\hat{\theta}&=\sqrt{EX^2-(EX)^2}\\
&=\sqrt{\frac{1}{n}\sum X_i^2-\bar{X}^2}\\
&=S_n
\end{align*}
\noindent\textbf{极大似然估计}\\
\begin{align*}
L(\mu,\theta)&=\Pi_{i=1}^{n}\frac{1}{\theta}e^{-\frac{X_i-\mu}{\theta}}\\
&=\frac{1}{\theta^n}e^{\sum_{i=1}^{n}-\frac{X_i-\mu}{\theta}}\\
lnL(\mu,\theta)&=ln\frac{1}{\theta^n}+\sum_{i=1}^{n}-\frac{X_i-\mu}{\theta}\\
\frac{\partial lnL(\mu,\theta)}{\partial \mu}&=\frac{n}{\theta} >0\\
\frac{\partial lnL(\mu,\theta)}{\partial \theta}&=-\frac{n}{\theta}+\sum_{i=1}^{n}\frac{X_i-\mu}{\theta^2}\\
&=-\frac{n}{\theta}+\frac{\sum_{i=1}^{n}X_i-n\mu}{\theta^2}
\therefore
\end{align*}
由于似然函数随$\mu$递增而递增,$\mu$最大为$\min_{i}X_i$.所以
\[
\hat{\mu}=\min_{i}X_i
\]
此时
\[
\hat{\theta}=\frac{\sum_{i=1}^{n}X_i-n\hat\mu}{n}=\bar{X}-\min_{i}X_i
\]
\section*{11}
无偏估计意味着$E[.]=\sigma^2$.\\
\begin{align*}
E[.]&=c\sum_{i=1}^{n-1}E(X_{i+1}-X_i)^2\\
&=c\sum_{i=1}^{n-1}\left[D(X_{i+1}-X_i)+(E(X_{i+1}-X_i))^2\right]\\
&=c(n-1)2\sigma^2\\
&=\sigma^2\\
\therefore c&=\frac{1}{2(n-1)}
\end{align*}
\section*{12}
\begin{align*}
L(\sigma^2)&=\Pi_{i=1}^{n}\frac{1}{\sqrt{2\pi\sigma^2}}e^{-\frac{(X_i-\mu)^2}{2\sigma^2}}\\
lnL(\sigma^2)&=\sum_{i=1}^{n}-\frac{(X_i-\mu)^2}{2\sigma^2}+ln\frac{1}{\sqrt{2\pi\sigma^2}}\\
\frac{dlnL(\sigma^2)}{d\sigma^2}&=\sum_{i=1}^{n}\frac{(x-\mu)^2}{2\sigma^4}+\sqrt{2\pi \sigma^2}(-\frac{1}{2\pi\sigma^2})\frac{\sqrt{2\pi}}{2\sqrt{\sigma^2}}\\
&=\sum_{i=1}^{n}\frac{(x-\mu)^2}{2\sigma^4}-\frac{1}{2\sigma^2}\\
&\text{令}\frac{dlnL(\sigma^2)}{d\sigma^2}=0\\
&\frac{\sum_{i=1}^{n}(X_i-\mu)}{2\sigma^2}=\frac{n}{2\sigma^2}\\
\hat{\sigma^2}&=\frac{1}{n}\sum_{i=1}^{n}(X_i-\mu)^2
\end{align*}
所以极大似然估计为$\hat{\sigma^2}=\frac{1}{n}\sum_{i=1}^{n}(X_i-\mu)^2$.\\
标准正态分布的k阶原点矩为\[
\begin{cases}
	(k-1)!!&\text{当k为偶}\\
	0&\text{当k为奇数}
\end{cases}\]
\begin{align*}
E\hat{\sigma^2}&=E(X_i-\mu)^2\\
&=\sigma^2E(\frac{X_i-\mu}{\sigma})^2\\
&=\sigma^2
\end{align*}
\begin{align*}
M(\hat{\sigma^2},\sigma^2)&=D(\hat{\sigma^2})+(E\hat{\sigma^2}-\sigma^2)^2\\
&=\frac{1}{n^2}\sum_{i=1}^{n}D(X_i-\mu)^2\\
&=\frac{1}{n}\left[E(X_i-\mu)^4-(E(X_i-\mu)^2)^2\right]\\
&=\frac{1}{n}\sigma^4\left[E(\frac{X_i-\mu}{\sigma})^4-(E(\frac{X_i-\mu}{\sigma})^2)^2\right]\\
&=\frac{2\sigma^4}{n}
\end{align*}
\begin{align*}
&ES^2\\
&=\frac{1}{n-1}\sum_{i=1}^{n}E(X_i-\bar{X})^2\\
&=\frac{1}{n-1}\sum_{i=1}^{n}D(X_i-\bar{X})\\
&=\frac{1}{n-1}\sum_{i=1}^{n}D(-\frac{1}{n}X_1......+\frac{n-1}{n}X_i-\frac{1}{n}X_{i+1}-...-\frac{1}{n}X_n)\\
&=\frac{1}{n-1}\sum_{i=1}^{n}\frac{n-1}{n}\sigma^2\\
&=\sigma^2\\
\\
\\
&M(S^2,\sigma^2)\\
&=E(S^2-\sigma^2)^2\\
&=(ES^2-\sigma^2)^2+DS^2\\
&=DS^2\\
&\because  \frac{(n-1)S^2}{\sigma^2}~~\sim~\chi^2(n-1)\\
&\therefore \frac{(n-1)^2}{\sigma^4}D(S^2)=D(\chi^2(n-1))=2n-2\\
&\therefore DS^2=\frac{2\sigma^4}{n-1}\\
&\therefore M(S^2,\sigma^2)=\frac{2\sigma^4}{n-1}\\
\\
\\
&\because M(\hat{\sigma^2},\sigma^2)=\frac{2\sigma^4}{n}\\
&~~~~M(S^2,\sigma^2)=\frac{2\sigma^4}{n-1}\\
&\therefore M(S^2,\sigma^2)>M(\hat{\sigma^2},\sigma^2)
\end{align*}
\section*{13}
\begin{align*}
\frac{\bar{X}-\mu}{\sigma/\sqrt{n}}&~\sim N(0,1)\\
P(Y_1\le Y\le Y_2)&=1-\alpha\\
P(0\le Y\le Y_2)&=\frac{1-\alpha}{2}\\
Y_2&=u_{\frac{\alpha}{2}}\\
-\frac{u_{\frac{\alpha}{2}}\sigma}{\sqrt{n}}&\le \bar{X}-\mu\le \frac{u_{\frac{\alpha}{2}}\sigma}{\sqrt{n}}\\
\bar{X}-\frac{u_{\frac{\alpha}{2}}\sigma}{\sqrt{n}}&\le \mu \le \bar{X}+\frac{u_{\frac{\alpha}{2}}\sigma}{\sqrt{n}}\\
\frac{2u_{\frac{\alpha}{2}}\sigma}{\sqrt{n}}&\le L\\
n&\ge \frac{4u_{\frac{\alpha}{2}}\sigma}{\sqrt{n}}
\end{align*}
\section*{14}
\begin{align*}
&n=9\\
&\frac{\bar{X}-\mu}{S/\sqrt{n}}~~\sim~t(n-1)\\
&P\{-\lambda\le \frac{\bar{X}-\mu}{S/\sqrt{n}}\le \lambda\}=95\%\\
&\lambda=t_{0.025}=2.3060\\
&S^2=0.33\\
&S=0.5745\\
&\bar{X}=6\\
&-2.3060\le \frac{6-\mu}{0.5745/3}\le 2.3060\\
&5.5584\le \mu\le 6.4416
\end{align*}

\section*{18}
大样本法,当$n$足够大时,根据中心极限定理,有
\[
\frac{\sum_{i=1}^{n}X_i-n\mu}{\sqrt{n}\sigma}~\to~N(0,1)
\]
\begin{align*}
&\frac{n\bar{X}-n\lambda}{\sqrt{n}\sqrt{\lambda}}~\to~N(0,1)\\
&\bar{X}~\to~\lambda  \text{~~when }n~\to \infty\\
&\frac{\bar{X}-\lambda}{\sqrt{\frac{\bar{X}}{n}}}~\to~N(0,1)\\
&P(X_1\le \frac{\bar{X}-\lambda}{\sqrt{\frac{\bar{X}}{n}}} \le X_2)=1-\alpha\\
&-u_{\frac{\alpha}{2}}\le\frac{\bar{X}-\lambda}{\sqrt{\frac{\bar{X}}{n}}}\le u_{\frac{\alpha}{2}}\\
&\bar{X}-u_{\frac{\alpha}{2}}\sqrt{\frac{\bar{X}}{n}}\le \lambda\le\bar{X}+u_{\frac{\alpha}{2}}\sqrt{\frac{\bar{X}}{n}}
\end{align*}
\end{document}