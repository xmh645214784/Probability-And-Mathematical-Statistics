\documentclass[a4paper,twocolumn]{ctexart}
\usepackage{ctex}
\usepackage{amsmath}


\graphicspath{{./figures/}}
%首行缩进两字符 利用\indent \noindent进行控制
\usepackage{indentfirst}
\setlength{\parindent}{2em}

%算法包
\usepackage{caption}
\usepackage{algorithm}
\usepackage{algorithmic}

%页边距包
\usepackage{geometry}
\geometry {left=2.0cm ,right=2.0cm,top=2.5cm,bottom=2.5cm}

%枚举
\usepackage{enumerate}

%代码
\usepackage{listings}

%算法input output
\renewcommand{\algorithmicrequire}{\textbf{Input:}} % Use Input in the format of Algorithm
\renewcommand{\algorithmicensure}{\textbf{Output:}} % Use Output in the format of Algorithm
% for fig without caption: #1: width/size; #2: fig file
\newcommand{\fignocaption}[2]{
	\begin{figure}[htp]
		\centering
		\includegraphics[#1]{#2}
	\end{figure}
}
% for fig with caption: #1: width/size; #2: fig file; #3: fig caption
\newcommand{\fig}[3]{
	\begin{figure}[htp]
		\centering
		\includegraphics[#1]{#2}
		\caption[labelInTOC]{#3}
	\end{figure}
}
%数学符号
\usepackage{amssymb}
\title{Probability And Mathmatical Statistics\\Homework 3}
\author{151220131 谢旻晖}
\date{}
\begin{document}
\maketitle
\section*{习题一 36}
设单位设立$n$条外线,则$P(\text{分机不被占线})=\sum_{i=0}^{n}\binom{100}{i}p^i(1-p)^{100-i}$,where $p=5\%$,上式大于90\%,解出最小的$n$为8.
\section*{习题二 7}
\noindent(1)\\
\indent $\frac{1}{\binom{8}{4}}=1.43\%$\\
\noindent (2)\\
\indent 记(1)中的概率为$p$,猜对的概率为$\binom{10}{3}p^3(1-p)^{7}=0.0316\%$,概率十分小,应该确实有分辨能力。
\section*{习题二 8}
$X$--$P(2.5)$\\
\noindent(1)\\
\[
P(X\ge4)=1-P(X\le3)=0.2424
\]
\noindent(2)\\
\[
\arg \max_{i\in \mathbb{N}}P(X=i)=2
\]
\[
P(X=2)=0.2565
\]
\noindent(3)\\
\[
\arg \min_{n\in \mathbb{N}}P(X\le n)\le 90\%
\]
\indent 解得$n=5$
\section*{习题二 19}
\noindent(1)\\
\begin{center}
\begin{tabular}{cccccc}
\hline
\hline
X&-4&-1&0&1&8\\
P&$\frac{1}{8}$&$\frac{1}{4}$&$\frac{1}{8}$&$\frac{1}{6}$&$\frac{1}{3}$\\
\hline
\end{tabular}
\end{center}

\noindent(2)\\
\begin{center}
	\begin{tabular}{ccccc}
		\hline
		\hline
		X&0&$\frac{1}{4}$&4&16\\
		P&$\frac{1}{8}$&$\frac{5}{12}$&$\frac{1}{8}$&$\frac{1}{3}$\\
		\hline
	\end{tabular}
\end{center}

\noindent(3)\\
\begin{center}
	\begin{tabular}{cccc}
		\hline
		\hline
		X&$-\frac{\sqrt{2}}{2}$&0&$\frac{\sqrt{2}}{2}$\\
		P&$\frac{1}{4}$&$\frac{7}{12}$&$\frac{1}{6}$\\
		\hline
	\end{tabular}
\end{center}

\section*{习题三 3}
\begin{align*}
&P(X+Y=2)\\
&=P(X=1,Y=1\cup X=2,Y=0 \cup X=3,Y=-1)\\
&=P(X=1,Y=1)+P(X=2,Y=0)+P(X=3,Y=-1)\\
&=\frac{1}{16}+\frac{1}{16}+\frac{1}{32}\\
&=\frac{5}{32}
\end{align*}

\section*{2.负二项分布}

\begin{align*}
&P(X=i)\\
&=P(\text{前i-1次出现k-1个正面,第i次是正面})\\
&=P(\text{前i-1次出现k-1个正面})P(\text{第i次是正面})\\
&=\binom{i-1}{k-1}p^{k-1}(1-p)^{i-k}*p\\
&=\binom{i-1}{k-1}p^k(1-p)^{i-k}
\end{align*}


\section*{3.蓄水池抽样}
\noindent 设数据流中数据的个数为$n$.\\
\noindent (1)\\
记事件$A_i$为第i个数据到来时发生了替换,有$P(A_i)=\frac{1}{i}$.\\
\begin{align*}
&P(\text{第i个数据最终留在了内存中})\\
&=P(A_i~\overline{A_{i+1}}~\overline{A_{i+2}}\ldots\overline{A_{n}})\\
&=\frac{1}{i}\frac{i}{i+1}\frac{i+1}{i+2}\ldots\frac{n-1}{n}\\
&=\frac{1}{n}
\end{align*}
上面的推导显示出了任何一个数据最终留在内存中的概率是一样的,与在数据流中的次序无关。算法是有效的。\\
\noindent(2)\\
记事件$A_i$为第i个数据到来时发生了替换,有$P(A_i)=\frac{1}{2}$.\\
\begin{align*}
&P(\text{第i个数据最终留在了内存中})\\
&=P(A_i~\overline{A_{i+1}}~\overline{A_{i+2}}\ldots\overline{A_{n}})\\
&=\frac{1}{2}\frac{1}{2}\frac{1}{2}\ldots\frac{1}{2}\text{~~~(n-i+1个2)}\\
&=\frac{1}{2^{n-i+1}}
\end{align*}

\section*{4.}
\[
E=\frac{6}{\pi^2}\sum_{i=1}^{\infty}\frac{1}{i}
\]
调和级数$\sum_{i=1}^{\infty}\frac{1}{i}$发散,所以期望不存在.
\section*{5.}
\indent 可以发现这样的事实,每一轮我们都从人群中剔除了一个“人”:以两种情况考虑,如果选中的是一个人的双手,那这个人自成环以后就再也不会成环,相当于这个人被剔除了;如果选中的是两个人的分别一只手,我们就将这两个人打包为一个“人”,也相当于剔除了一个人。\\
\indent 在这种条件下,我们发现在每一轮中形成环当且仅当选中的是一个“人”的双手。\\
\indent 定义指示器随机变量$A_i$为
$$A_i=
\begin{cases}
1 &\text{第i轮选择生成了新的环}\\
0 &\text{第i轮选择没有新的环生成}
\end{cases}
$$
由上面的讨论$P(A_i=1)=\frac{n-i+1}{\binom{2n-2i+2}{2}}$,分母为从$2n-2i+2$只手中选2只牵手,分子为选到的是一个“人”的两只手。\\
\indent 那么牵手形成环的期望值为
\begin{align*}
&~E\\
&=E\left(\sum_{i=1}^{n}A_i\right)\\
&=\sum_{i=1}^{n}E(A_i)\\
&=\sum_{i=1}^{n}P(A_i=1)*1\\
&=\sum_{i=1}^{n}\frac{n-i+1}{\binom{2n-2i+2}{2}}\\
&=\sum_{i=1}^{n}\frac{1}{2n-2i+1}
\end{align*}
\section*{6.Jensen Inequality}
首先证明对于下凸函数有对于任意$x_1,x_2\ldots x_n$,$p_1,p_2\ldots p_n$,$0\le p_i\le1$,且$\sum_{i=1}^{n}p_i=1$.有\\
\[
f(\sum_{i=1}^{n}p_ix_i)\le \sum_{i=1}^{n}p_if(x_i)
\]
使用数学归纳法证明,对$n$进行归纳.\\
当$n=2$时,trivial.\\
假设$n=k$时,有对于任意$x_1,x_2\ldots x_n$,$p_1,p_2\ldots p_k$,$0\le p_i\le1$,且$\sum_{i=1}^{k}p_i=1$.有
\[
f(\sum_{i=1}^{k}p_ix_i)\le \sum_{i=1}^{k}p_if(x_i)
\]
当$n=k+1$时,
\begin{align*}
&f(\sum_{i=1}^{k+1}p_ix_i)\\
&=f\left(\left(1-p_{k+1}\right)\frac{\sum_{i=1}^{k}p_ix_i}{1-p_{k+1}}+p_{k+1}x_{k+1}\right)\\
&\le \left(1-p_{k+1}\right)f\left(\frac{\sum_{i=1}^{k}p_ix_i}{1-p_{k+1}}\right)+p_{k+1}f\left(x_{k+1}\right)\\
&\le(1-p_{k+1})\sum_{i=1}^{k}\frac{p_i}{1-p_{k+1}}f(x_i)+p_{k+1}f\left(x_{k+1}\right)~~~~\text{由假设}\\
&=\sum_{i=1}^{k+1}p_if\left(x_i\right)
\end{align*}
归纳法证明成功。\\
再证明命题.\\
设X对应$n$个离散型随机变量$x_1,x_2\ldots x_n$,对应的概率分别为$p_1,p_2\ldots p_n$.由概率完备性$\sum_{i=1}^{n}p_i=1$\\
由期望的定义$E(X)=\sum_{i=1}^{n}p_ix_i$
\begin{align*}
&f\left(E(X)\right)\\
&=f\left(\sum_{i=1}^{n}p_ix_i\right)\\
&\le \sum_{i=1}^{n}p_if\left(x_i\right)\\
&=E\left(f(x_i)\right)
\end{align*}
证毕。
\end{document}